\mytopic{How to calculate $\pi$ with rice}

Draw a square and a quarter circle in it on a paper and throw a hand full of rice on it. Done!

\vskip 0.5em
Well almost done. Now you count how many grains of rice landed inside the square and how many landed inside the quarter circle.
The ratio of these two numbers approximates the ratio of the two areas ($\pi\times$r\textsuperscript{\kern 1pt2}/\,4) / r\textsuperscript{\kern 1pt2} = $\pi$/4.
Multiply the ratio by 4 and you get $\pi$, well it is an approximation. If you are not happy with the accuracy just use more rice.

\vskip 0.5em
This is one example of the Monte Carlo method. The key is to avoid computing a complicated formula by using random samples in a clever way.

% change: quarter circle 2x, fix the formula!
