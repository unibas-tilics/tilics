\mytopic{No Mercy: The Year 2038 Problem}

The date of \fbox{Jan 1, 1970} is to UNIX developpers what year 0 is for
Christians, just that in UNIX time you count in seconds, not in days
as we do with calendars.

\vskip 0.5em
\noindent Now here is a problem:
\begin{itemize}
        \labelsep 0.5em%
        \labelwidth \leftmargini%
        \addtolength\labelwidth{-\labelsep}
        % \listparindent 5em%   
        \itemindent 1.5em%\leftmargini
        %\advance\leftmargini 1.5em
\item in UNIX time, two billion seconds \emph{after} \fbox{Jan 1, 1970} is \fbox{Jan 19, 2038}
\item in UNIX time, two billion seconds \emph{before} \fbox{Jan 1, 1970} is \fbox{Dec 13, 1901}
\end{itemize}

\vskip 0.5em
\noindent Many UNIX systems can hold time values only up to four billions, and
then values start to ``wrap around''. This means that on \fbox{Jan 19, 2038},
these UNIX systems will think it is the year 1901!

The clock is now ticking for finding all these old UNIX systems and to
update their software, giving them more bits for storing {\em UNIX}
time. For sure, some systems will be missed and some systems will be
too old to be fixed.\\
{\em Real} time will have no mercy \ldots

