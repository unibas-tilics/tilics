\mytopic{What if humans had 8 fingers instead of 10?}

Whenever we are dealing with numbers in our daily lives, we use the so called \emph{decimal system}. This means that our entire counting system is based on the number 10. A given number, for example 17, actually means {\em ``1 time the number 10, plus 7 times the number 1''}. It’s the same as counting with your fingers. Once you reached 10, you remember the result and start over again with one finger. 

\vskip 0.5em
Now what if we had 8 fingers instead? Our counting system might be based on the number 8. This would be called \emph{octal system}. Our decimal number 17, also called \fbox{\tt 17\raisebox{-0.2em}{\footnotesize dec}} to prevent confusion, would instead be written as \fbox{\tt 21\raisebox{-0.2em}{\footnotesize oct}}, which actually stands for {\em ``2 times the number 8, plus 1 time the number 1''}.

