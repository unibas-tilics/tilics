\mytopic{Moore's Law}

In 1965, the technological progress lead Gordon E. Moore to formulate his rule of thumb that became famous under the name ``Moore's Law''. It states that the production costs of microchips used in electronic circuits would fall, leading to smaller and more powerful microchips at a lower overall cost. He estimated, the number of components on a single chip to double roughly every 12 months.

Initially the increase in the number of transistors used in a single chip was responsible for this exponential growth. Later, the downscaling of the components themselves became the driving force.

However, it is unlikely that technological progress will follow Moore's Law forever. As components get smaller and microchips get denser, the manufacturing process will inevitably hit physical limits at some point.

