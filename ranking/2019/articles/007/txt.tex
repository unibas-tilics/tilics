\mytopic{Zero is not null }

The null value has a special meaning in computer science. It is used to declare that a value is not yet defined or known.
The specialty of null is that although it shows us that a value is missing, it itself is a value. So, it is possible to compare two
nonexistent values. Suppose we have data about a patient;
it does make a difference whether the patient has no disease (0) or is not yet diagnosed (null).

