Studying Computer Science is like taking a deep dive into an endless
ocean of ideas, concepts, and tools. It is a vast ecosystem where
everything is interconnected, where exploring one thing opens doors to
a plethora of other, equally interesting, things. It is easy to get
lost in this wide landscape, even more so if it is not one's main
field of expertise.

This is why Computer Science students of the University of Basel set
out to identify the essential concepts and ideas they discovered
during their studies. The intention was to go beyond the scope of
traditional textbooks and also include the pecularities that are not
formally part of a usual Computer Science curriculum. Finally, the
goal was to present the fruit of this work to people outside of our
field.

\newpage

\vspace*{2cm}
The creation of this booklet was organized in the form of a Master's
level seminar, which had its first iteration during the spring
semester 2019. We have listed the participants on the inside back cover.

The result is
{\bf 2\raisebox{0.5em}{\tiny\artcount} Things I Learned in Computer Science},
a collection of short illustrated articles. It is available in the
form of this small booklet, and online on the companion website
$[$1$]$. Due to the success and the positive feedback, we will continue
this work and are now planning for a next edition of the seminar.

\vskip 0.5em\noindent\raggedleft\hfill
We hope that you enjoy these info bytes\\
\hfill and that you pass them on to your friends!

\raggedright
\vspace*{\fill}\noindent
{\small\tt $[$1$]$ https://tilics.dmi.unibas.ch/}
